%\documentclass[draft]{dune}
\documentclass[final]{dune}
\usepackage[utf8]{inputenc}
\usepackage{amssymb}
\usepackage{changepage}
\usepackage{xspace}
\usepackage{indentfirst}
%\usepackage{siunitx}
\input{units.tex}
\usepackage[hidelinks]{hyperref}
\input{defs.tex}
\input{glossary.tex}
\title{DAPHNE Review}
\author{ }
\date{ }

\begin{document}
\maketitle

\begin{abstract}
DAPHNE is the front-end board for the Single Phase PDS at DUNE experiment in FERMILAB. DAPHNE is designed to digitize analog signals coming from the Photon Detection Sensors in the Cold Electronics inside of the TPC cryostat. It is the first instrumentation module after the TPC interface, on the warm side. The next ideas will help to understand the architecture and the way to control it.

%If the following information has wrong content you are free to modify or delete it.
\rm
\begin{adjustwidth}{2cm}{2cm} 

\end{adjustwidth}
\end{abstract}

\section{DAPHNE Requirements}
\label{sec:requirements}
The next constraints are considered mandatory for the design:
\begin{itemize}
    \item Signal-to-noise > 4 			(SP-PDS-14)
    \item < 1 us time resolution 		(SP-FD-4)
    \item < 1 kHZ dark noise rate 		(SP-PDS-15)
    \item 2000 PE dynamic range 		(SP-PDS-16)
\end{itemize}
Moreover, the next design constraints are considered for granularity, space in the cryostat inteface, and in order to warrantee signal description:
\begin{itemize}
\item 40 Channel granularity
\item 14 bit resolution
\item 65 MSPS
\item High SNR
\item Power supply for the active ganging
\item Increase the bandwidth of the output (4.8Gb/s)
\end{itemize}
\newpage
“The quantitative requirements for the system are driven by many FD level specifications that affect signal size sensitivity, signal to noise (S/N), timing resolution, event size and data transfer limits from the DAQ, power needs and dissipation limits, channel density and channel count, and cost.” 



\section{DAPHNE Interfaces}
\label{sec:interfaces}

The FEB has a direct connection with 4 subsystems: Cold Electronics, DAQ, Slow control, and the cryostat interface.

\subsection{COLD ELECTRONICS:}

The digitization module consist of a differential to single converter, the AFE, and a readout control interface. The differential treatment of the signal between the cold electronics and DAPHNE is mandatory for the signal integrity. The constraints for this treatment are the distance to the cryostat, the impedance of the carrier cables and the frequency of the expected pulses.  A signal preconditioning has been done before into the cold environment by the active ganging. 

DAPHNE supplies a bias voltage (56V), and a digitally controlled trim voltage (0V-4.096) to the cold electronics. Another digital voltage reference for the analog signal conversion is a voltage reference that moves the reference into the AFE. All these signals are driven by DACs attached to the FPGA.

Actually the cables between DAPHNE and the cold electronics interface are HDMI cables supporting 4 analog channels each, for a total of 10 HDMI cables for one FEB. 

\subsection{SLOW CONTROL: }

\subsubsection{Fast Ethernet.}
The Slow Control Connection consists of a Fast Ethernet Connection drived by the ST microcontroller.
The output of the link is an SFP transceiver, allowing optical fiber, RJ45, and a TCP/IP support.
This interface should Handle flags in the power requirements of the Board, the monitoring of the XADCs on the FPGA, and the General Error Handling of the FEB and Cold Electronics. 
The FEB implements a server where a register matrix is refreshed in the microcontroller and in the FPGA.

An error handling protocol must be designed.

\subsubsection{Debugging Pins}
There is a set of debugging pins to perform an event check in case of differences between the simulation and the hardware results.

\subsection{DAQ:}

\subsubsection{SFP x2 - Full Mode x2}
Daphne counts on two SFP transceivers for data streaming. These are driven directly by the GTP transceivers in FPGA and are 4.8Gb/s capable in streaming mode. At DUNE final assembly only one of these will be used for data streaming. The use of two digital streaming enables us to perform loop corrections at the development level, and at the calibration rounds. 

\subsubsection{Timing Recovery CDR}
DAQ should provide a system clock of 62.5MHz for DAPHNE and should be connected at the CDR port (Optical fiber). 

\subsubsection{LEMO Input}
Nevertheless DAPHNE is prepared to work with an external clock input source at the LEMO connection. This connection will be used during the tests in absence of a system triggering clock, or with different rates at the streaming output like ICEBERG. 

\subsubsection{LEMO Output}
The LEMO output allows us to sync other modules sharing the clock of the FPGA. It could be the LEMO input clock or the one of the inner programmable clock on the FPGA. A control loop can be runned to check the clock integrity of some modules or signals.




%\section{DAPHNE Board Schematic}
\label{sec:board}

\section{Power Consumption}
\label{sec:interfaces}

Using an FPGA power estimator tool, we have estimated that our FPGA will consume about 3.5 Watts and that our AFEs at about 150mW/channel and at 40 channels will consume about 6 Watts. Our cooling fan is estimated to consume about 1.6W of power as well. We estimate that our board will consume about 26 W total. 

\subsection{Power Distribution: }

To generate the bias voltage, a Cockcroft-Walton (CW) Generator is used. +/-10V is generated from the secondary winding of the main power transformer on DAPHNE. This voltage is then fed to the CW Generator . It is possible to choose what bias voltage will be generated. +34V, +45V, +55V, +66V, and +77V can all be chosen.
\\*
\\* Voltages generated for the FPGA are described below in the Power-on Sequence subsection. Voltages generated for the various op-amps are shown in the power distribution diagram. 

\subsection{Power Connectors: }

The DAPHNE board power distribution has similarities to the Mu2E CRV Front End Board. One of the similarities is that they both have a power barrel connector for bench top testing. 
But DAPHNE also needs a different power connector. There are some options. A bulkhead connector, wires, mounted male connector and female mate could be used similar to the alternative to HDMI connectors described above. 
Another option is to use a molex through hole connector with a center pin of 0.156”. Such connectors have been used on boards for CDMS and Mu2E.

\subsection{Cooling Fan: }

For a cooling fan we have decided to use a brushless axial ball bearing flow 5V fan, EFB0405VHD-F00, by Delta Electronics.  We will use a 3 pin connector to connect to the fan: +5V, GND, and Frequency Generator (FG). The FG produces a square wave signal with a frequency that is proportional to the fan speed. This signal is sent to the microcontroller and can allow for detection of low supply voltage or blocked airflow. The use of this function will be optional, and programming will be needed within the microcontroller. 
This fan also includes a surface mount fuse rated for 500mA, 0154.500DR. For now, the plan is to keep the fan constantly turned on once plugged in. This fan has a life expectancy of about 70,000 hours (7.99 years) under continuous operation at 104° F (40°C). 


\subsection{FPGA Power-on Sequence: }

The recommended power-on sequence to achieve minimum current draw for the GTP transceivers is:
\\*
\\*VCCINT (1V) – Internal Supply Voltage
\\*VCCBRAM (1V) – Supply Voltage for block RAM memories
\\*VMGTAVCC (1V) – Analog Supply for the GTP circuits
\\*VMGTAVTT (1.2V) – Analog Supply for the GTP termination circuits
\\*VCCAUX (1.8V) – Auxiliary Supply Voltage
\\*VCCO (2.5V, 3.3V)– Output Drivers Supply Voltage For I/O Banks
\\*
\\*Both VMGTAVCC and VCCINT can be ramped simultaneously. The recommended
power-off sequence is the reverse of the power-on sequence to achieve minimum current draw. The reason for using a linear regulator for  MGTAVCC is because it is very sensitive to noise.
\\*
\\*In the NEXYS video schematic, which uses the same FPGA, an input of 12V is used. In our case we will be using an input of about 8V (same as from Sten’s Mu2E FEB power scheme). 
\\*
\\*The three devices that power the FPGA are the:
 ADP2325 which supplies 1V(5A) and 2.5V(500mA)
LTC3868 which supplies 1.8V(5A) and 3.3V(3.5A)
TPS7A91 which supplies 1V(1A) and 1.2V(1A). 
The 1V is for VMGTAVCC and the  1.2V is for VMGTAVTT, both required for the GTP transceiver. VMGTAVCC is the Analog supply voltage for the GTP transmitter and receiver circuits. VMGTAVTT is the Analog supply voltage for the GTP transmitter and receiver termination circuits. 
\\*
\\*No external device is used to create this sequence. 
The sequencing is done using the enable pins on the voltage regulators. And all of these voltages except VMGTAVTT and VMGTAVCC have a ramp time of 8ms. 
 

\subsection{Cold Electronics Voltage: }

For the cold electronics voltage we have decided to go with +3V. To generate this voltage both a buck converter, LTC3624, and linear regulator, TPS7A701, are used. The buck converter will take +5.5V coming from the power transformer and convert it into +3.3V. This +3.3V will not be used for any other part of the board or FPGA. The linear regulator will then convert the voltage to +3V. 
Both the LTC3624 and TPS7A701 are adjustable so if a different cold electronics voltage was decided upon other than +3V then it could done by changing feed back resistors. 


\subsection{Artix-7 vs Kintex-7: }

To switch to the Kintex-7 FPGA with faster transceiver capabilities (9.6 Gb/s) will take some considerable work and cost. The cost of a Kintex-7 that can supply the 9.6 Gb/s speed is about \$1.5k. This is about 7 times as much as the Artix-7. The new FPGA would then have to be studied and all pins on the schematic would have to be rerouted. The power supply would also have to be redesigned to suit a more powerful FPGA. Also due to the faster speed of the device a different kind of pcb material may have to be used instead of FR-4. It is possible that material from Rogers Corp. may be needed to meet the needs of the 9.6 Gb/s transmissions. 

\subsection{HDMI vs other Connector: }

A possible scheme to switch to a bulkhead connector would be placing the bulkhead connectors along the wall of the 1U/2U box. In place of using 10 hdmi connectors, 10 bulkhead connectors would be used. The bulkhead would then need to be wired to a different female connector. Both ends of the wires will need to crimped onto both connectors. Then the female connector can mate with a male 90 degree connector mounted on the board. This way the board is easily detached to the 1U/2U box.

\section{Remarks}
\label{sec:Remarks}

After checking the status of the DAPHNE interfaces there are still open points to be defined:

\begin{enumerate}
    \item Floating or non-floating power supply for the cold amplifier.
    \item External power supply.
    \item To final definition of DAPHNE cable a detailed scheme of PDS-Flange is needed to check for available space.
    \item Error handling in agreement with other groups and Slow Control
\end{enumerate}

%\bibliographystyle{plain}\it
%\bibliography{bibfile}

\end{document}
